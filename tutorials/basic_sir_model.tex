% Options for packages loaded elsewhere
\PassOptionsToPackage{unicode}{hyperref}
\PassOptionsToPackage{hyphens}{url}
\documentclass[
  a4paper,
]{article}
\usepackage{xcolor}
\usepackage[margin=1in]{geometry}
\usepackage{amsmath,amssymb}
\setcounter{secnumdepth}{-\maxdimen} % remove section numbering
\usepackage{iftex}
\ifPDFTeX
  \usepackage[T1]{fontenc}
  \usepackage[utf8]{inputenc}
  \usepackage{textcomp} % provide euro and other symbols
\else % if luatex or xetex
  \usepackage{unicode-math} % this also loads fontspec
  \defaultfontfeatures{Scale=MatchLowercase}
  \defaultfontfeatures[\rmfamily]{Ligatures=TeX,Scale=1}
\fi
\usepackage{lmodern}
\ifPDFTeX\else
  % xetex/luatex font selection
\fi
% Use upquote if available, for straight quotes in verbatim environments
\IfFileExists{upquote.sty}{\usepackage{upquote}}{}
\IfFileExists{microtype.sty}{% use microtype if available
  \usepackage[]{microtype}
  \UseMicrotypeSet[protrusion]{basicmath} % disable protrusion for tt fonts
}{}
\makeatletter
\@ifundefined{KOMAClassName}{% if non-KOMA class
  \IfFileExists{parskip.sty}{%
    \usepackage{parskip}
  }{% else
    \setlength{\parindent}{0pt}
    \setlength{\parskip}{6pt plus 2pt minus 1pt}}
}{% if KOMA class
  \KOMAoptions{parskip=half}}
\makeatother
\usepackage{color}
\usepackage{fancyvrb}
\newcommand{\VerbBar}{|}
\newcommand{\VERB}{\Verb[commandchars=\\\{\}]}
\DefineVerbatimEnvironment{Highlighting}{Verbatim}{commandchars=\\\{\}}
% Add ',fontsize=\small' for more characters per line
\usepackage{framed}
\definecolor{shadecolor}{RGB}{248,248,248}
\newenvironment{Shaded}{\begin{snugshade}}{\end{snugshade}}
\newcommand{\AlertTok}[1]{\textcolor[rgb]{0.94,0.16,0.16}{#1}}
\newcommand{\AnnotationTok}[1]{\textcolor[rgb]{0.56,0.35,0.01}{\textbf{\textit{#1}}}}
\newcommand{\AttributeTok}[1]{\textcolor[rgb]{0.13,0.29,0.53}{#1}}
\newcommand{\BaseNTok}[1]{\textcolor[rgb]{0.00,0.00,0.81}{#1}}
\newcommand{\BuiltInTok}[1]{#1}
\newcommand{\CharTok}[1]{\textcolor[rgb]{0.31,0.60,0.02}{#1}}
\newcommand{\CommentTok}[1]{\textcolor[rgb]{0.56,0.35,0.01}{\textit{#1}}}
\newcommand{\CommentVarTok}[1]{\textcolor[rgb]{0.56,0.35,0.01}{\textbf{\textit{#1}}}}
\newcommand{\ConstantTok}[1]{\textcolor[rgb]{0.56,0.35,0.01}{#1}}
\newcommand{\ControlFlowTok}[1]{\textcolor[rgb]{0.13,0.29,0.53}{\textbf{#1}}}
\newcommand{\DataTypeTok}[1]{\textcolor[rgb]{0.13,0.29,0.53}{#1}}
\newcommand{\DecValTok}[1]{\textcolor[rgb]{0.00,0.00,0.81}{#1}}
\newcommand{\DocumentationTok}[1]{\textcolor[rgb]{0.56,0.35,0.01}{\textbf{\textit{#1}}}}
\newcommand{\ErrorTok}[1]{\textcolor[rgb]{0.64,0.00,0.00}{\textbf{#1}}}
\newcommand{\ExtensionTok}[1]{#1}
\newcommand{\FloatTok}[1]{\textcolor[rgb]{0.00,0.00,0.81}{#1}}
\newcommand{\FunctionTok}[1]{\textcolor[rgb]{0.13,0.29,0.53}{\textbf{#1}}}
\newcommand{\ImportTok}[1]{#1}
\newcommand{\InformationTok}[1]{\textcolor[rgb]{0.56,0.35,0.01}{\textbf{\textit{#1}}}}
\newcommand{\KeywordTok}[1]{\textcolor[rgb]{0.13,0.29,0.53}{\textbf{#1}}}
\newcommand{\NormalTok}[1]{#1}
\newcommand{\OperatorTok}[1]{\textcolor[rgb]{0.81,0.36,0.00}{\textbf{#1}}}
\newcommand{\OtherTok}[1]{\textcolor[rgb]{0.56,0.35,0.01}{#1}}
\newcommand{\PreprocessorTok}[1]{\textcolor[rgb]{0.56,0.35,0.01}{\textit{#1}}}
\newcommand{\RegionMarkerTok}[1]{#1}
\newcommand{\SpecialCharTok}[1]{\textcolor[rgb]{0.81,0.36,0.00}{\textbf{#1}}}
\newcommand{\SpecialStringTok}[1]{\textcolor[rgb]{0.31,0.60,0.02}{#1}}
\newcommand{\StringTok}[1]{\textcolor[rgb]{0.31,0.60,0.02}{#1}}
\newcommand{\VariableTok}[1]{\textcolor[rgb]{0.00,0.00,0.00}{#1}}
\newcommand{\VerbatimStringTok}[1]{\textcolor[rgb]{0.31,0.60,0.02}{#1}}
\newcommand{\WarningTok}[1]{\textcolor[rgb]{0.56,0.35,0.01}{\textbf{\textit{#1}}}}
\usepackage{graphicx}
\makeatletter
\newsavebox\pandoc@box
\newcommand*\pandocbounded[1]{% scales image to fit in text height/width
  \sbox\pandoc@box{#1}%
  \Gscale@div\@tempa{\textheight}{\dimexpr\ht\pandoc@box+\dp\pandoc@box\relax}%
  \Gscale@div\@tempb{\linewidth}{\wd\pandoc@box}%
  \ifdim\@tempb\p@<\@tempa\p@\let\@tempa\@tempb\fi% select the smaller of both
  \ifdim\@tempa\p@<\p@\scalebox{\@tempa}{\usebox\pandoc@box}%
  \else\usebox{\pandoc@box}%
  \fi%
}
% Set default figure placement to htbp
\def\fps@figure{htbp}
\makeatother
\setlength{\emergencystretch}{3em} % prevent overfull lines
\providecommand{\tightlist}{%
  \setlength{\itemsep}{0pt}\setlength{\parskip}{0pt}}
\usepackage{booktabs}
\usepackage{longtable}
\usepackage{array}
\usepackage{multirow}
\usepackage{wrapfig}
\usepackage{float}
\usepackage{colortbl}
\usepackage{pdflscape}
\usepackage{tabu}
\usepackage{threeparttable}
\usepackage{threeparttablex}
\usepackage[normalem]{ulem}
\usepackage{makecell}
\usepackage{xcolor}
\usepackage{bookmark}
\IfFileExists{xurl.sty}{\usepackage{xurl}}{} % add URL line breaks if available
\urlstyle{same}
\hypersetup{
  pdftitle={Basic SIR Model},
  pdfauthor={Jamie Prentice},
  hidelinks,
  pdfcreator={LaTeX via pandoc}}

\title{Basic SIR Model}
\author{Jamie Prentice}
\date{2026-02-05}

\begin{document}
\maketitle

\subsection{Load libraries and source
files}\label{load-libraries-and-source-files}

This pulls in all the necessary libraries and source files

\begin{Shaded}
\begin{Highlighting}[]
\FunctionTok{source}\NormalTok{(}\StringTok{"libraries.R"}\NormalTok{)}
\FunctionTok{source}\NormalTok{(}\StringTok{"source\_files.R"}\NormalTok{)}
\end{Highlighting}
\end{Shaded}

\subsection{Generating a params list}\label{generating-a-params-list}

The first step is to generate a params list that contains the basic
information, with some messages to note exactly what it's using. All
elements in \texttt{make\_parameters()} have default values, so these
are just to see what you can add.

The params list might not fully do everything we want, but as it is just
a basic R list, we can easily modify it afterwards.

\begin{Shaded}
\begin{Highlighting}[]
\NormalTok{params }\OtherTok{\textless{}{-}} \FunctionTok{make\_parameters}\NormalTok{(}
    \AttributeTok{model\_type =} \StringTok{"SIR"}\NormalTok{,  }\CommentTok{\# a Susceptible{-}Infected{-}Removed model}
    \AttributeTok{setup =} \StringTok{"single"}\NormalTok{,    }\CommentTok{\# a single group}
    \AttributeTok{use\_traits =} \StringTok{"none"}\NormalTok{, }\CommentTok{\# no genetic effects applied to traits}
    \AttributeTok{sim\_new\_data =} \StringTok{"r"}   \CommentTok{\# simulate in R (rather than in BICI)}
\NormalTok{)}
\end{Highlighting}
\end{Shaded}

This provides a short summary of the params

\begin{Shaded}
\begin{Highlighting}[]
\FunctionTok{summarise\_params}\NormalTok{(params)}
\end{Highlighting}
\end{Shaded}

\begin{verbatim}
## Simulating new SIR data via R
\end{verbatim}

\begin{verbatim}
## - Demography is:
##  10 sires, 20 dams, 500 progeny (530 total)
##  1 group (group size 500)
\end{verbatim}

\begin{verbatim}
## - Individual Effects on:
##  none
\end{verbatim}

\begin{verbatim}
## - Sigma_G:
\end{verbatim}

\begin{verbatim}
##     sus inf tol
## sus   0   0   0
## inf   0   0   0
## tol   0   0   0
\end{verbatim}

\begin{verbatim}
## - Fixed Effects are:
##  Trial = 'none', Donor = 'none', TxD = 'none', Weight = 'none'
\end{verbatim}

\begin{verbatim}
## - Estimated R0:
##  5
\end{verbatim}

\begin{verbatim}
## - Running MCMC with:
##  10,000 updates, 10,000 samples, 0.2 burnin, 16 chains
## - BICI script file:
##  'datasets/testing/data/scen-1-1.bici'
## - Results file:
##  'datasets/testing/results/scen-1-1.rds'
\end{verbatim}

\subsection{Generate pedigree and
popn}\label{generate-pedigree-and-popn}

Next we take the \texttt{params} file and use it to generate a
population with groups, traits, weights, and fixed effects applied to
the phenotypes. Note that you these are piped into each other, but you
can save the intermediate result at any point to examine what's
happening (e.g.~for bug fixing).

First \texttt{make\_pedigree()} creates a population with the
appropriate structure (this may be a balanced population with given
numbers of sires, dams per sire, and progeny per dam, it may copy an
existing structure, or you can just set your own structure)

\begin{Shaded}
\begin{Highlighting}[]
\NormalTok{popn }\OtherTok{\textless{}{-}} \FunctionTok{make\_pedigree}\NormalTok{(params) }\SpecialCharTok{|\textgreater{}}
    \FunctionTok{set\_groups}\NormalTok{(params) }\SpecialCharTok{|\textgreater{}}
    \FunctionTok{set\_traits}\NormalTok{(params) }\SpecialCharTok{|\textgreater{}}
    \FunctionTok{set\_weights}\NormalTok{(params) }\SpecialCharTok{|\textgreater{}}
    \FunctionTok{apply\_fixed\_effects}\NormalTok{(params)}
\end{Highlighting}
\end{Shaded}

It's worth taking a look at the \texttt{popn} file to see what it
includes. The columns are:

\begin{Shaded}
\begin{Highlighting}[]
\FunctionTok{names}\NormalTok{(popn)}
\end{Highlighting}
\end{Shaded}

\begin{verbatim}
##  [1] "id"     "sire"   "dam"    "sdp"    "trial"  "group"  "weight" "donor" 
##  [9] "GE"     "sus_g"  "inf_g"  "tol_g"  "sus_e"  "inf_e"  "tol_e"  "sus"   
## [17] "inf"    "tol"
\end{verbatim}

Where:

\begin{itemize}
\tightlist
\item
  \texttt{id}: the unique ID of the individual. Sires come first, then
  dams, then progeny.
\item
  \texttt{sire} and \texttt{dam} of the individual (together with
  \texttt{id} this makes a pedigree).
\item
  \texttt{sdp}: whether the individual is a sire, dam, or progeny.
\item
  The \texttt{weight} in g at start of experiment.
\item
  Which \texttt{trial} the individual was assigned to.
\item
  Which \texttt{group} the individual was assigned to.
\item
  \texttt{donor}: if the individual was inoculated (1) or not (0).
\item
  \texttt{GE}: a small tank dependent group effect.
\item
  \texttt{sus\_g}, \texttt{inf\_g}, \texttt{tol\_g}, \texttt{sus\_e},
  \texttt{inf\_e}, \texttt{tol\_e}: genetic and environmental values.
  These are normally distributed with mean 0.
\item
  \texttt{sus}, \texttt{inf}, \texttt{tol}, \texttt{lat}, \texttt{det}:
  phenotypic values. These are log-normally distributed and incorporate
  any fixed effects.
\end{itemize}

A typical progeny looks like:

\begin{Shaded}
\begin{Highlighting}[]
\NormalTok{popn[sdp }\SpecialCharTok{==} \StringTok{"progeny"}\NormalTok{][}\DecValTok{1}\NormalTok{] }\SpecialCharTok{|\textgreater{}}
\NormalTok{    knitr}\SpecialCharTok{::}\FunctionTok{kable}\NormalTok{(}\StringTok{"html"}\NormalTok{)}
\end{Highlighting}
\end{Shaded}

id

sire

dam

sdp

trial

group

weight

donor

GE

sus\_g

inf\_g

tol\_g

sus\_e

inf\_e

tol\_e

sus

inf

tol

31

1

11

progeny

1

1

50

1

1

0

0

0

0

0

0

1

1

1

\subsection{Simulate and plot
epidemic}\label{simulate-and-plot-epidemic}

We now pass the population file \texttt{popn} and the parameters
\texttt{params} to \texttt{simulate\_epidemic()}, which will run the
appropriate model and return a \emph{new} \texttt{popn} file with event
times, final status, which generation infective they were, and the
individual responsible for their infection. You can generate multiple
realisations of the epidemic from the same inputs if you save the
\texttt{popn} output each time.

\begin{Shaded}
\begin{Highlighting}[]
\FunctionTok{tic}\NormalTok{(); popn }\OtherTok{\textless{}{-}} \FunctionTok{simulate\_epidemic}\NormalTok{(popn, params); }\FunctionTok{toc}\NormalTok{()}
\end{Highlighting}
\end{Shaded}

\begin{verbatim}
## 5.169 sec elapsed
\end{verbatim}

We can also use the generation value to calculate \(R_0\), which is the
mean across all groups of the ratio of secondary to primary infectives.
Also it's useful to see how long the epidemic took until it completed
(since this information is necessary for BICI to use).

\begin{Shaded}
\begin{Highlighting}[]
\NormalTok{params}\SpecialCharTok{$}\NormalTok{estimated\_R0 }\OtherTok{\textless{}{-}} \FunctionTok{get\_R0}\NormalTok{(popn)}
\end{Highlighting}
\end{Shaded}

\begin{verbatim}
## R0 estimate: 7.4
\end{verbatim}

\begin{Shaded}
\begin{Highlighting}[]
\NormalTok{params}\SpecialCharTok{$}\NormalTok{tmax }\OtherTok{\textless{}{-}} \FunctionTok{get\_tmax}\NormalTok{(popn, params)}

\FunctionTok{message}\NormalTok{(}\StringTok{"Tmax = ["}\NormalTok{, }\FunctionTok{str\_flatten\_comma}\NormalTok{(}\FunctionTok{round}\NormalTok{(params}\SpecialCharTok{$}\NormalTok{tmax, }\DecValTok{1}\NormalTok{)), }\StringTok{"]"}\NormalTok{)}
\end{Highlighting}
\end{Shaded}

\begin{verbatim}
## Tmax = [78.6]
\end{verbatim}

The progeny we had before now looks like this, with the additional
columns:

\begin{itemize}
\tightlist
\item
  \texttt{Tinf}: the infection time (\(t:S\rightarrow E\)). In actual
  data this is 0 for inoculated individuals, and missing for all others.
\item
  \texttt{Tinc}: the incubation time (\(t:E\rightarrow I\)), missing in
  the data.
\item
  \texttt{Tsym}: the time of symptoms (\(t:I\rightarrow D\)), present in
  data.
\item
  \texttt{Tdeath}: the time of death (\(t:D\rightarrow R\)), present in
  data.
\item
  \texttt{status}: which compartment the individual is in at the end of
  the epidemic (should only be \texttt{S} or \texttt{R})
\item
  \texttt{generation}: if an individual is a primary, secondary,
  tertiary etc. infective.
\item
  \texttt{infected\_by}: the id of the individual responsible for this
  infection (0 for inoculated individuals).
\item
  \texttt{parasites} \texttt{TRUE} if an individual was infected. In
  actual data the sensitivity is not 100\%.
\end{itemize}

\begin{Shaded}
\begin{Highlighting}[]
\NormalTok{popn[sdp }\SpecialCharTok{==} \StringTok{"progeny"}\NormalTok{,}
\NormalTok{     .(id, donor, Tinf, Tdeath, status, generation, infected\_by)][}\DecValTok{1}\SpecialCharTok{:}\DecValTok{10}\NormalTok{] }\SpecialCharTok{|\textgreater{}}
\NormalTok{    knitr}\SpecialCharTok{::}\FunctionTok{kable}\NormalTok{(}\StringTok{"html"}\NormalTok{)}
\end{Highlighting}
\end{Shaded}

id

donor

Tinf

Tdeath

status

generation

infected\_by

31

1

0.00000

22.381736

R

1

0

32

1

0.00000

10.104750

R

1

0

33

1

0.00000

0.144765

R

1

0

34

1

0.00000

3.853585

R

1

0

35

1

0.00000

3.714627

R

1

0

36

0

23.51434

39.132388

R

NA

111

37

0

29.20198

41.491427

R

NA

57

38

0

13.60348

15.852539

R

NA

21

39

0

20.38388

45.431064

R

NA

61

40

0

24.00863

52.788157

R

NA

104

\subsection{Plot the epidemic}\label{plot-the-epidemic}

We can take a look at the time trajectory

\begin{Shaded}
\begin{Highlighting}[]
\NormalTok{p1a }\OtherTok{\textless{}{-}} \FunctionTok{plot\_model}\NormalTok{(popn, params)}
\end{Highlighting}
\end{Shaded}

\begin{figure}
\centering
\pandocbounded{\includegraphics[keepaspectratio]{basic_sir_model_files/figure-latex/unnamed-chunk-10-1.pdf}}
\caption{Time trajectory of SEIDR model}
\end{figure}

It's also important to look at the Kaplan-Meier survival plot.

\begin{Shaded}
\begin{Highlighting}[]
\NormalTok{p1b }\OtherTok{\textless{}{-}} \FunctionTok{basic\_km}\NormalTok{(popn, params)}
\end{Highlighting}
\end{Shaded}

\begin{figure}
\centering
\pandocbounded{\includegraphics[keepaspectratio]{basic_sir_model_files/figure-latex/unnamed-chunk-11-1.pdf}}
\caption{KM plot of SEIDR model, showing the proportion of each family
surviving over time.}
\end{figure}

\subsection{Add in genetic variance}\label{add-in-genetic-variance}

\begin{Shaded}
\begin{Highlighting}[]
\NormalTok{params2 }\OtherTok{\textless{}{-}} \FunctionTok{make\_parameters}\NormalTok{(}
    \AttributeTok{model\_type =} \StringTok{"SIR"}\NormalTok{, }\CommentTok{\# a Susceptible{-}Infected{-}Removed model}
    \AttributeTok{setup =} \StringTok{"single"}\NormalTok{,   }\CommentTok{\# a single group}
    \AttributeTok{use\_traits =} \StringTok{"sit"}\NormalTok{, }\CommentTok{\# no genetic effects applied to traits}
    \AttributeTok{vars =} \FunctionTok{list}\NormalTok{(}\AttributeTok{sus =} \FloatTok{0.5}\NormalTok{, }\AttributeTok{inf =} \DecValTok{1}\NormalTok{, }\AttributeTok{tol =} \FloatTok{0.2}\NormalTok{),}
    \AttributeTok{cors =} \FunctionTok{list}\NormalTok{(}\AttributeTok{si =} \FloatTok{0.3}\NormalTok{, }\AttributeTok{st =} \SpecialCharTok{{-}}\FloatTok{0.2}\NormalTok{, }\AttributeTok{it =} \FloatTok{0.2}\NormalTok{),}
    \AttributeTok{sim\_new\_data =} \StringTok{"r"}  \CommentTok{\# simulate in R (rather than in BICI)}
\NormalTok{)}
\end{Highlighting}
\end{Shaded}

A quick look at the summary

\begin{Shaded}
\begin{Highlighting}[]
\FunctionTok{summarise\_params}\NormalTok{(params2)}
\end{Highlighting}
\end{Shaded}

\begin{verbatim}
## Simulating new SIR data via R
\end{verbatim}

\begin{verbatim}
## - Demography is:
##  10 sires, 20 dams, 500 progeny (530 total)
##  1 group (group size 500)
\end{verbatim}

\begin{verbatim}
## - Individual Effects on:
##  sus, inf, tol
\end{verbatim}

\begin{verbatim}
## - Sigma_G:
\end{verbatim}

\begin{verbatim}
##      sus inf  tol
## sus  0.5 0.3 -0.2
## inf  0.3 1.0  0.2
## tol -0.2 0.2  0.2
\end{verbatim}

\begin{verbatim}
## - Fixed Effects are:
##  Trial = 'none', Donor = 'none', TxD = 'none', Weight = 'none'
\end{verbatim}

\begin{verbatim}
## - Estimated R0:
##  5
\end{verbatim}

\begin{verbatim}
## - Running MCMC with:
##  10,000 updates, 10,000 samples, 0.2 burnin, 16 chains
## - BICI script file:
##  'datasets/testing/data/scen-1-1.bici'
## - Results file:
##  'datasets/testing/results/scen-1-1.rds'
\end{verbatim}

Now rebuild the population

\begin{Shaded}
\begin{Highlighting}[]
\NormalTok{popn2 }\OtherTok{\textless{}{-}} \FunctionTok{make\_pedigree}\NormalTok{(params2) }\SpecialCharTok{|\textgreater{}}
    \FunctionTok{set\_groups}\NormalTok{(params2) }\SpecialCharTok{|\textgreater{}}
    \FunctionTok{set\_traits}\NormalTok{(params2) }\SpecialCharTok{|\textgreater{}}
    \FunctionTok{set\_weights}\NormalTok{(params2) }\SpecialCharTok{|\textgreater{}}
    \FunctionTok{apply\_fixed\_effects}\NormalTok{(params2)}
\end{Highlighting}
\end{Shaded}

If we look at some of the individuals, we can see that they now have
individual values for their genotypes, and different phenotypes

\begin{Shaded}
\begin{Highlighting}[]
\NormalTok{popn2[sdp }\SpecialCharTok{==} \StringTok{"progeny"}\NormalTok{, }\FunctionTok{map}\NormalTok{(.SD, round, }\DecValTok{2}\NormalTok{),}
\NormalTok{     .SDcols }\OtherTok{=} \FunctionTok{str\_subset}\NormalTok{(}\FunctionTok{names}\NormalTok{(popn2), }\StringTok{"id|sus|inf|tol"}\NormalTok{)][}\DecValTok{1}\SpecialCharTok{:}\DecValTok{5}\NormalTok{] }\SpecialCharTok{|\textgreater{}}
\NormalTok{    knitr}\SpecialCharTok{::}\FunctionTok{kable}\NormalTok{(}\StringTok{"html"}\NormalTok{)}
\end{Highlighting}
\end{Shaded}

id

sus\_g

inf\_g

tol\_g

sus\_e

inf\_e

tol\_e

sus

inf

tol

31

-0.27

1.09

-0.94

-1.15

0.06

-0.01

0.24

3.15

0.39

32

-0.25

-0.24

-0.57

-0.79

0.45

-0.92

0.36

1.23

0.23

33

-0.41

-2.22

-0.57

-0.42

-1.07

-0.81

0.44

0.04

0.25

34

-0.10

-0.24

-1.66

0.51

1.12

-1.23

1.50

2.41

0.06

35

-0.74

-1.34

-0.49

-1.18

0.42

0.41

0.15

0.40

0.92

We can now run a simulation with the new setup

\begin{Shaded}
\begin{Highlighting}[]
\NormalTok{popn2 }\OtherTok{\textless{}{-}} \FunctionTok{simulate\_epidemic}\NormalTok{(popn2, params2)}
\FunctionTok{get\_R0}\NormalTok{(popn2)}
\end{Highlighting}
\end{Shaded}

\begin{verbatim}
## [1] 0.4
\end{verbatim}

\subsection{Plot the epidemic}\label{plot-the-epidemic-1}

And plot the results, noting a much faster epidemic

\begin{Shaded}
\begin{Highlighting}[]
\NormalTok{p2a }\OtherTok{\textless{}{-}} \FunctionTok{plot\_model}\NormalTok{(popn2, params2)}
\end{Highlighting}
\end{Shaded}

\pandocbounded{\includegraphics[keepaspectratio]{basic_sir_model_files/figure-latex/unnamed-chunk-17-1.pdf}}

\begin{Shaded}
\begin{Highlighting}[]
\NormalTok{p2b }\OtherTok{\textless{}{-}} \FunctionTok{basic\_km}\NormalTok{(popn2, params2)}
\end{Highlighting}
\end{Shaded}

\pandocbounded{\includegraphics[keepaspectratio]{basic_sir_model_files/figure-latex/unnamed-chunk-18-1.pdf}}

Now we can add in some fixed effects.

\begin{Shaded}
\begin{Highlighting}[]
\NormalTok{params3 }\OtherTok{\textless{}{-}} \FunctionTok{make\_parameters}\NormalTok{(}
    \AttributeTok{model\_type =} \StringTok{"SIR"}\NormalTok{, }\CommentTok{\# a Susceptible{-}Infected{-}Removed model}
    \AttributeTok{setup =} \StringTok{"multiple"}\NormalTok{, }\CommentTok{\# a single group}
    \AttributeTok{use\_traits =} \StringTok{"sit"}\NormalTok{, }\CommentTok{\# no genetic effects applied to traits}
    \AttributeTok{vars =} \FunctionTok{list}\NormalTok{(}\AttributeTok{sus =} \FloatTok{0.5}\NormalTok{, }\AttributeTok{inf =} \DecValTok{1}\NormalTok{, }\AttributeTok{tol =} \FloatTok{0.2}\NormalTok{),}
    \AttributeTok{cors =} \FunctionTok{list}\NormalTok{(}\AttributeTok{si =} \FloatTok{0.3}\NormalTok{, }\AttributeTok{st =} \SpecialCharTok{{-}}\FloatTok{0.2}\NormalTok{, }\AttributeTok{it =} \FloatTok{0.2}\NormalTok{),}
    \AttributeTok{donor\_fe =} \StringTok{"it"}\NormalTok{,    }\CommentTok{\# donor FE on inf, tol}
    \AttributeTok{weight\_fe =} \StringTok{"sit"}\NormalTok{,  }\CommentTok{\# weight FE on sus, inf, tol}
    \AttributeTok{sim\_new\_data =} \StringTok{"r"}  \CommentTok{\# simulate in R (rather than in BICI)}
\NormalTok{)}
\FunctionTok{summarise\_params}\NormalTok{(params3)}
\end{Highlighting}
\end{Shaded}

There's one more step here, we need to specify the FE values we want to
use. It's just a matrix containing all the combinations.

\begin{Shaded}
\begin{Highlighting}[]
\NormalTok{params3}\SpecialCharTok{$}\NormalTok{fe\_vals[}\StringTok{"donor"}\NormalTok{, }\StringTok{"inf"}\NormalTok{] }\OtherTok{\textless{}{-}} \DecValTok{1}
\NormalTok{params3}\SpecialCharTok{$}\NormalTok{fe\_vals[}\StringTok{"donor"}\NormalTok{, }\StringTok{"tol"}\NormalTok{] }\OtherTok{\textless{}{-}} \SpecialCharTok{{-}}\DecValTok{1}
\NormalTok{params3}\SpecialCharTok{$}\NormalTok{fe\_vals[}\StringTok{"weight"}\NormalTok{, }\FunctionTok{c}\NormalTok{(}\StringTok{"sus"}\NormalTok{, }\StringTok{"inf"}\NormalTok{, }\StringTok{"tol"}\NormalTok{)] }\OtherTok{\textless{}{-}} \FunctionTok{c}\NormalTok{(}\SpecialCharTok{{-}}\FloatTok{0.5}\NormalTok{, }\FloatTok{0.5}\NormalTok{, }\FloatTok{0.5}\NormalTok{)}
\NormalTok{params3}\SpecialCharTok{$}\NormalTok{fe\_vals[, params3}\SpecialCharTok{$}\NormalTok{model\_traits]}
\end{Highlighting}
\end{Shaded}

\begin{verbatim}
##          traits
## fe         sus inf  tol
##   trial    0.0 0.0  0.0
##   donor    0.0 1.0 -1.0
##   txd      0.0 0.0  0.0
##   weight  -0.5 0.5  0.5
##   weight1  0.0 0.0  0.0
##   weight2  0.0 0.0  0.0
\end{verbatim}

And generate and simulate

\begin{Shaded}
\begin{Highlighting}[]
\NormalTok{popn3 }\OtherTok{\textless{}{-}} \FunctionTok{make\_pedigree}\NormalTok{(params3) }\SpecialCharTok{|\textgreater{}}
    \FunctionTok{set\_groups}\NormalTok{(params3) }\SpecialCharTok{|\textgreater{}}
    \FunctionTok{set\_traits}\NormalTok{(params3) }\SpecialCharTok{|\textgreater{}}
    \FunctionTok{set\_weights}\NormalTok{(params3) }\SpecialCharTok{|\textgreater{}}
    \FunctionTok{apply\_fixed\_effects}\NormalTok{(params3) }\SpecialCharTok{|\textgreater{}}
    \FunctionTok{simulate\_epidemic}\NormalTok{(params3)}
\end{Highlighting}
\end{Shaded}

And plot

\begin{Shaded}
\begin{Highlighting}[]
\NormalTok{p2a }\OtherTok{\textless{}{-}} \FunctionTok{plot\_model}\NormalTok{(popn3, params3)}
\end{Highlighting}
\end{Shaded}

\pandocbounded{\includegraphics[keepaspectratio]{basic_sir_model_files/figure-latex/unnamed-chunk-22-1.pdf}}

\begin{Shaded}
\begin{Highlighting}[]
\NormalTok{p2b }\OtherTok{\textless{}{-}} \FunctionTok{basic\_km}\NormalTok{(popn3, params3)}
\end{Highlighting}
\end{Shaded}

\pandocbounded{\includegraphics[keepaspectratio]{basic_sir_model_files/figure-latex/unnamed-chunk-23-1.pdf}}

\end{document}
